\section{Preliminaries}

Analogous to Euler's classical forward difference expansion, we propose a multiplicative formulation of function interpolation based on a newly defined \textit{ratio operator}. The central identity we aim to establish is:

\begin{align}
f(a + x) = \prod_{k=0}^{\infty} \left( \rho^k f(a) \right)^{\binom{x}{k}},
\end{align}

where $\rho^k f(a)$ denotes the $k$-th order ratio operator applied to the function $f$ at the base point $a$. The operator is defined recursively as follows:

\begin{align}
\rho f(a) &= \frac{f(a+1)}{f(a)}, \\
\rho^2 f(a) &= \rho\left(\rho f(a)\right) = \frac{\rho f(a+1)}{\rho f(a)}, \\
\rho^k f(a) &= \rho\left( \rho^{k-1} f(a) \right) = \frac{\rho^{k-1} f(a+1)}{\rho^{k-1} f(a)}.
\end{align}

This operator serves as a multiplicative counterpart to the forward difference $\Delta^k f(a)$, and its structure mirrors the behavior of derivatives in a ratio-based discrete system. 

To prove the identity, we consider the discrete sequence of function values:

\begin{align*}
\gamma = \{ f(a),\ f(a+1),\ f(a+2),\ f(a+3),\ \dots,\ f(a+n) \}
\end{align*}

From this set, we compute the successive ratio operators $\rho^k f(a)$ for all $k \in \mathbb{N}$. Let us define the index set:

\[
k = \{0, 1, 2, 3, \dots, k\}
\]

and the corresponding set of operator values:

\[
\rho_k = \{ \rho^0 f(a),\ \rho^1 f(a),\ \rho^2 f(a),\ \rho^3 f(a),\ \dots,\ \rho^k f(a) \}
\]

By observing the structure formed through repeated ratios, and inspired by the binomial coefficients from Pascal's triangle, we find that:

\[
f(a + n - 1) = \rho^0 f(a)^{\binom{n-1}{0}} \cdot \rho^1 f(a)^{\binom{n-1}{1}} \cdot \rho^2 f(a)^{\binom{n-1}{2}} \cdots \rho^k f(a)^{\binom{n-1}{k}}
\]

which can be compactly written as:

\begin{align}
f(a + n - 1) &= \prod_{i=0}^{k} \left( \rho^i f(a) \right)^{\binom{n-1}{i}} \\
f(a + n)     &= \prod_{i=0}^{k} \left( \rho^i f(a) \right)^{\binom{n}{i}}
\end{align}

This forms the foundation for a multiplicative analogue of Newton’s interpolation, where the exponents follow binomial growth and the base values evolve via successive multiplicative differences.

Thus our proof of $(1)$ is completed by the derivation of $(6)$.
