\section{On Transformations}

Consider Euler's classical forward difference formula:

\begin{align}
f(a + x) = \sum_{k=0}^{\infty} \binom{x}{k} \Delta^k f(a)
\end{align}

Let us define the original forward difference set as:

\[
\Delta = \{ \Delta^0 f(a),\ \Delta^1 f(a),\ \Delta^2 f(a),\ \dots,\ \Delta^k f(a) \}
\]

Now, define a transformed set $\Delta'$, which we refer to as the **first-order Pascal transform** of $\Delta$, denoted:

\[
P_1(\Delta) = \Delta' = \left\{
\Delta^0 f(a),\ \Delta^0 f(a) + \Delta^1 f(a),\ \Delta^1 f(a) + \Delta^2 f(a),\ \dots,\ \Delta^{k-1} f(a) + \Delta^k f(a),\ \Delta^k f(a)
\right\}
\]

This transformation resembles the additive behavior seen in Pascal’s triangle, shifting the structure of the differences to cumulative adjacent pairs.

We now aim to show the following identity:  
If

\[
f(a + x) = \sum_{k=0}^{\infty} \binom{x}{k} \Delta^k f(a), \quad \Delta^k f(a) \in \Delta
\]

then the cumulative sum satisfies:

\begin{align}
\sum_{x=0}^{n} f(a + x) = \sum_{k=0}^{\infty} \binom{n}{k} \Delta'^k f(a), \quad \Delta'^k f(a) \in \Delta'
\end{align}

This provides a discrete analogue of the integral of an interpolated function, where the difference terms are transformed via $P_1$. This structure hints at a broader algebra of transformations over difference sets and may lead to higher-order transforms $P_2, P_3, \dots$, each corresponding to deeper combinatorial structures.

To prove this transformation identity, we consider the forward differences of the function $f(x)$, denoted by the set $\gamma$, defined as:

\[
\gamma = \left\{
f(a),\
f(a+1) - f(a),\
f(a+2) - 2f(a+1) + f(a),\
\dots,\
\sum_{i=0}^{n} \binom{n-1}{i} f(a+i)(-1)^{n-i-1}
\right\}
\]

In general, the $k$-th term of this sequence is given by:

\begin{align}
\Delta^k f(a) = \sum_{i=0}^{k-1} \binom{k-1}{i} f(a+i)(-1)^{k-i-1}
\end{align}

Now, consider the cumulative sum sequence of the function:

\[
\left\{
f(a),\
f(a) + f(a+1),\
f(a) + f(a+1) + f(a+2),\
\dots,\
\sum_{i=0}^{n} f(a+i)
\right\}
\]

We denote the forward differences of this sequence by the set $\gamma'$. These differences can be partitioned into three subsets:

\begin{align}
\gamma' &= \gamma_1 \cup \gamma_2 \cup \gamma_3
\end{align}
\begin{align}
\gamma_1 &= \{ f(a) \} 
\end{align}
\begin{align}
\gamma_2 &= \left\{
f(a+1),\
f(a+2) - f(a+1),\
f(a+3) - 2f(a+2) + f(a+1),\
\dots,\
\sum_{i=0}^{n-1} \binom{n-1}{n-i-1} f(a+i+1)(-1)^{n-i-1}
\right\}
\end{align}
\begin{align}
\gamma_3 &= \left\{
\sum_{i=0}^{n} \binom{n-1}{i} f(a+i)(-1)^{n-i-1}
\right\}
\end{align}

Notably, the $k$-th term of the central subset $\gamma_2$ is given by the expression:

\begin{align}
\Delta'^k = \sum_{i=0}^{k-1} \binom{k-1}{i} f(a+i+1)(-1)^{k-i-1}
\end{align}

We now demonstrate the elegant identity:
\[
\Delta^k f(a) + \Delta^{k+1} f(a) = \Delta'^k f(a)
\]
where $\Delta'^k f(a)$ denotes the $k$-th term in the forward difference sequence of the cumulative sum $S(x) = \sum_{j=0}^x f(j)$. Assuming $k+1 \leq n$, we expand the terms explicitly.

First, consider the standard expressions for the $k$-th and $(k+1)$-th forward differences of $f$:

\begin{align}
\Delta^k f(a) &= \sum_{i=0}^{k-1} \binom{k-1}{i} f(a+i)(-1)^{k-i-1} \\
\Delta^{k+1} f(a) &= \sum_{i=0}^{k} \binom{k}{i} f(a+i)(-1)^{k-i}
\end{align}

Adding both expressions:

\begin{align}
\Delta^k f(a) + \Delta^{k+1} f(a) 
&= \sum_{i=0}^{k-1} \binom{k-1}{i} f(a+i)(-1)^{k-i-1}
+ \sum_{i=0}^{k} \binom{k}{i} f(a+i)(-1)^{k-i} \\
&= \sum_{i=1}^{k-1} \left[\binom{k-1}{i} - \binom{k}{i} \right] f(a+i)(-1)^{k-i-1}
+ f(a+k)
\end{align}

Using the Pascal identity:
\[
\binom{k-1}{i} - \binom{k}{i} = -\binom{k-1}{i-1}
\]

we get:

\begin{align}
\Delta^k f(a) + \Delta^{k+1} f(a) 
= \sum_{i=1}^{k-1} \left[ -\binom{k-1}{i-1} \right] f(a+i)(-1)^{k-i-1} + f(a+k)
= \sum_{i=1}^{k-1} \binom{k-1}{i-1} f(a+i)(-1)^{k-i} + f(a+k)
\end{align}

Changing the index in the summation:

\begin{align}
= \sum_{i=0}^{k-2} \binom{k-1}{i} f(a+i+1)(-1)^{k-i-1} + f(a+k)
= \sum_{i=0}^{k-1} \binom{k-1}{i} f(a+i+1)(-1)^{k-i-1}
\end{align}

This final expression is precisely the definition of $\Delta'^k f(a)$, hence:
\[
\Delta^k f(a) + \Delta^{k+1} f(a) = \Delta'^k f(a)
\]

This gives a relation between the central set $\gamma_2$ and the set $\gamma$, mirroring our first-order Pascal Transformation, and hence proving $(8)$.

Thus, let
\[
\underbrace{
P_k\big(
P_k\big(
\cdots
P_k\big(
\Delta
\big)
\cdots
\big)
\big)
}_{n \text{ times}}
= \Delta_k^n
\]
Then, we can write:

\begin{align}
\underbrace{
\sum_{x=0}^{n}
\sum_{x=0}^{n}
\cdots
\sum_{x=0}^{n}
}_{s \text{ times}}
f(a + x)
&= \left( \sum_{x=0}^{n} \right)^s f(a + x) \notag \\
&= \sum_{k=0}^{\infty} \binom{x}{k} \Delta_k f(a),
\quad \Delta_k f(a) \in \Delta_1^s
\end{align}

A similar identity for $(1)$ can be established if we redefine the transform $P_2(\Delta)$ to perform **multiplication** instead of **addition**, as in Pascal's triangle. That is, assume a set:
\[
A = \{ a_0, a_1, a_2, a_3, a_4, \dots, a_n \}
\]
Then define:
\[
P_2(A) = \{ a_0,\ a_1 \cdot a_0,\ a_2 \cdot a_1,\ \dots,\ a_n \cdot a_{n-1},\ a_n \}
\]

Now, we define another such identity. Let:
\[
\Delta_2 = \{ \rho^0 f(a),\ \rho^1 f(a),\ \dots,\ \rho^n f(a) \}
\]
Let the iterative application of the transform $P_2$ on $\Delta_2$, applied $s$ times, be denoted by $\Delta_2^s$. Then:

\begin{align}
\underbrace{
\prod_{x=0}^{n}
\prod_{x=0}^{n}
\cdots
\prod_{x=0}^{n}
}_{s \text{ times}}
f(a + x)
&= \left( \prod_{x=0}^{n} \right)^s f(a + x) \notag \\
&= \prod_{k=0}^{\infty} \binom{x}{k} \Delta_k f(a),
\quad \Delta_k f(a) \in \Delta_2^s
\end{align}
