\section{Factorial-like Functions and Tetrational Expansion}

$(7)$ does not expand the tetrational series and exhibits irregular behavior at certain points on the real line. To achieve a more accurate approximation of the factorial-like function, we instead utilize $(1)$, expressing the function as an infinite product.

Using these means one can obtain a product expression for the tetration function, where if $^k a$ is the tetration of $a$ as a base with a tower height of $k$, then one may find, using the product formula and methods in this paper that,

\begin{align}
^{n+1} a &= \prod_{t=0}^{\infty} \prod_{i=0}^{t} \left( {}^{i+1} a \right)^{(-1)^{t-i} \cdot \binom{t}{i} \cdot \binom{n}{t} }
\end{align}

The expression $(23)$ is a purely algebraical expression for the tetration function analogous to the binomial expansion for exponents.

Similarly for Factorial-like functions, one may use Pascal's second order transform and the product formula (both were discussed above in this paper) to conclude that,

\begin{align}
\prod_{i=0}^{n} (a+i) &= a \cdot \prod_{t=0}^{\infty} \prod_{i=0}^{t} (1+a+i)^{(-1)^{t-i} \cdot \binom{t}{i} \cdot \binom{n}{t+1}}
\end{align}

The expression $(24)$ is our approximation of the factorial function, however, this may misbehave at some points, as already known in interpolation theory.
